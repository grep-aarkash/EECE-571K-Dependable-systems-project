
\begin{abstract}

\noindent Autonomous Driving Systems(ADS) have seen an
immense increase in popularity recently and are expected to
become fully operational on normal roads in the near future. The
safety critical nature of autonomous driving systems (ADS) of
AVs makes quantifying the resilience properties of such systems
highly important. This helps in ensuring public and governmental
support for their adoption. The resilience of ADS, in general, has been
explored before but the effect on the reliability of ADS due to specific
components such as sensors present in ADS still remains to be
explored.

We make the following main contributions: 1. We define different
categories of failures which can occur in an ADS and 2. We do a
case study on one of the core sensors(optical sensors) in an ADS,
by defining its fault model and finding the failure rates of ADS
due to different faults in the fault model. We use CARLA, an
open source driving simulator, to conduct the experiments.

\end{abstract}
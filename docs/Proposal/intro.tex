\section{Introduction}

Fully automated self driving vehicles are poised to take the roads in near future. Partial self driving systems have already been deployed and tested for over a billion miles~\cite{teslasr},  providing valuable sensor data that is used to improve the performance of automated driving systems. The adoption of autonomous driving systems (ADS) on a large scale in real world is highly dependent on their reliability. Moreover, for the acceptance of deployment of such safety critical systems by wider public, makes it necessary to understand the resilience profiles of autonomous driving systems, identify and improve weaknesses in current designs and exhaustively test these fixes to increase public confidence.

% Add line above about what we gonna do
 
Autonomous Driving System are composed of multiple sub-systems, the major parts being the sensor arrays and the control systems. A wide range of sensors like RGB cameras, radars, lidars, sonar, GPS etc. are used to provide information about the environment to the control system of ADS, which usually uses a Deep Neural Network to find the best course of action and gives commands to actuators, to control the movement of the vehicle. Such a highly integrated systems introduce various concerns about the overall reliability of the system. Moreover as ADS are highly safety critical, regulations and standards impose stringent reliability requirements on their components. For example the SDC FIT rate (Failure-in-Time rate) for of SoC used for DNN inferencing specified by ISO-26262 is 10 i.e. ten failures in one billion hours of operation~\cite{guanpeng17sc}. These strict reliability requirements, demand that a reliability analysis be performed on integrated systems and vulnerabilities or faults that can propagate, from one sub-system to other sub-systems be identified and mitigation techniques applied to make the system resilient to such faults.

%Introduce Carla
Studying the autonomous driving system is hindered by resource intensive nature of training and testing solutions in real world scenarios. Dosovitskiy et al.~\cite{carla18corl} recently introduced CARLA, which is an open source simulator for training and validating autonomous driving platforms. It provides an urban environment setting from which data is captured using a flexible array of sensors and signals that are normally present in ADS, this simulated data can then be used to train and test different sub-systems of an ADS. As CARLA also provides sensor reading in addition to images captured by camera, it can be used to study the effect of a vulnerability present in a single component (for example the LIDAR sensor) on overall reliability of the whole ADS. CARLA operates in a server-client model. Server simulates the environment and generates readings for different sensors which are then provided to the client which acts as a driving agent (DA) taking different control decisions based on the provided sensor inputs. Separation of server and client functionality permits implementation of a wide range of DAs.

% Research statement
\emph{As per our knowledge, we are the first to attempt to find the failures caused by sensors in a self-driving car.}

%Paper Flow
Section 2 introduces the related work done in the resilience checking of self-driving cars and how our work is different from the existing literature. Section 3 introduces the background related to self-driving cars. This section talks about the different sensors which form the core components of the self-driving cars and then explains the indepth working of LIDAR, since finding failures due to LIDAR is the core contribution of our work.  Section 4 walks through the fault model and the fault-injection techniques we use for obtaining the failure rates of the sensors specifically LIDAR in our case. 
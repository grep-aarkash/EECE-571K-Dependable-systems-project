 \section{Discussion}
In order to perform the first set of experimentation, we initially chose LIDAR. However, simulating a LIDAR sensor is difficult since, the number of states in LIDAR are not finite. It is continuously rotating and obtaining new results and hence changing states. Hence, we decided to change our initial sensor and chose a camera instead.

The first fault-injection methodology though provides a good coverage incurs a lot of overhead. In order to optimize the injection process, we inject faults after every 10 frames. However, we observe that this reduces the coverage though providing us with better overhead. Hence, this becomes a question of trade-off between overhead and coverage. 

\subsection{Limitations of CARLA}
The degree to which CARLA models the real world autonomous driving systems and their interaction with the environment, is dependent on the simulation parameters and driving agents used for controlling the ADS. In real world autonomous driving systems an array of sensors is used, which captures a representation of the environment. This captured data is used by the controller to make driving decisions. CARLA provides the capability to simulate important sensors like cameras and LIDARs but some important sensors like Radar can't be simulated. Moreover the developed driving agents do not incorporate all the sensors available in CARLA. For example, the three driving agents studied by Dosovitskiy et al.~\cite{Dosovitskiy17} do not use LIDAR and rely only on camera output to take driving decisions. Using only camera sensors is not a realistic modeling of real world driving systems. 

There are no open source driving agents currently available that use sensors other than camera so in our study the driving agent we chose also uses only camera sensors. As our focus is on faults in camera sensors, this provides us good insight into reliability of ADS due to these faults. Presence of other sensor in the system may lead to masking of these due to other sensors, which is a more realistic representation of a real world system. Studying the effect sensor faults in presence of other sensors, is the next logical step after studying the reliability of ADS due to individual sensor faults.  

\section{Future work}
We intend to setup our test environment in a distributed environment and observe the failure rates in that context. In that case, we don't look at one car individually but we have multiple cars who are supposed to work with each other.


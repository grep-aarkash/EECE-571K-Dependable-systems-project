\section{Evaluation}
In this section we present the fault model for the sensors. We further evaluate the failure rates caused due to the faults in the sensor. We address the following research questions:
\begin{enumerate}
	\item \textbf{RQ1}
	What all types of faults can exist in advanced and core sensors of Autonomous vehicles such as the optical sensor.?
	\item \textbf{RQ2}
	What subset of the above faults cause the most amount of system damage i.e system failure?
	\item \textbf{RQ3}
	What is the failure rate of the faults causing the most amount of damage?
	\item \textbf{RQ4}
	What mitigation strategies can we use for preventing those failure rates due to the particular sensor faults in Autonomous vehicles?
\end{enumerate}
\vskip 0.2in
We answer the above research questions. 
\vskip 0.2in
\textbf{RQ1: What all types of faults exist in advance and core sensors in an autonomous vehicle.}
Though there have been lot of studies which look at the individual sensors. These studies also performed an indepth analysis of the faults which can be present in a sensors such as high-powered camera \cite{5530865}, \cite{inproceedings}. However, none of the previous work categorizes the fault model into hardware faults and image faults as shown in fig. 2. We not only constructed a fault model for the high-powered sensors, but we chose the faults which are most likely to occur in case of a self driving car. The previous work by Bresolin et al. \cite{inproceedings} performs a fault diagnosis of hybrid systems. However, their fault model is limited as they only consider the hardware faults. 

\subsection{Hardware faults}
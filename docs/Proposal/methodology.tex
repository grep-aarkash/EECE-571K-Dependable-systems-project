\section{Systematic Reliability Analysis}
%Here what we propose?
We propose development of a systematic method to find vulnerabilities in components of an ADS, instead of using standard ad-hoc approaches~\cite{jha18dsn}~\cite{jha18art}, which don't provide good coverage for resilience assessment. The main objective of this project will be to develop a technique that can detect all vulnerabilities in sub-components that lead to degradation of reliability of whole system. This is an unexplored area and the first question in this research will be to find out, if this method is feasible. This involves formulating of fault models for individual components of an ADS. After development of fault models, individual components will be manually tested for vulnerabilities to test the validity of these fault models. This testing will be generalized and automated later by developing a tool that systematically finds all the vulnerabilities specified by fault model of a component. The information collected using these individual fault models will then be used to find the vulnerabilities in the integrated ADS system.

CARLA will be used as a test platform for development and testing of our approach as it provides a convenient way to instrument and modify the readings of different sensors, as well as the behavior of driver agents that controls the autonomous vehicle. We intend to start our vulnerability analysis, by analyzing the effect on reliability of the system due to fault in one of the sensors. This will help in finding out the validity and effectiveness of our approach on a smaller scale so that any adjustment required in our research plan, can be carried out in the initial stages of the project. 

Current approaches that analyze resilience of ADS systems on a holistic level perform fault injections to find vulnerabilities in the system which is quite resource intensive. Details about such approaches is provided in section \ref{related_work}. Our proposed methodology will forgo such ad hoc approaches and test different components of an ADS in a systematic way by modeling its behavior and using formal methods to validate its correctness and finding any unexpected behavior that introduces vulnerabilities in the ADS system. 